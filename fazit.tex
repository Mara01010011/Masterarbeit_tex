% zusammenfassen, was untersucht wurde
Diese Arbeit beschäftigte sich mit dem Konstruieren eines Schuhs für den von Softrobotics hergestellen humanoiden Roboter NAO. Ziel war es, das bisher bereits vielseitig angewendete, aber noch nicht an Roboterfüßen verwendete MAP zu testen, eine Teststrecke mit der Möglichkeit Magneten anzubringen und verschiedene Winkel einzustellen zu bauen und die Stabilität des NAO zu erhöhen. Des Weiteren sollten die Auswirkungen der neuen Lauffläche auf den Roboter getestet werden. 

% was waren die Ergebnisse
Zusammenfassend lässt sich, wie in Kapitel \ref{gleichgewicht} beschrieben, einen Zusammenhang zwischen erhöhter Stabilität und Magneten erkennen. Die Prozentanteile die verwendet wurden, unterscheiden sich sehr im Anteil von CIP und Gewicht. Die Reaktion auf das Magnetfeld waren für höhere Prozente ersichtlich, während der Originalschuh kaum Reaktion zeigte. Der Gang mit $20\,\%$ war ungewöhnlich stabil, sowohl mit als auch ohne Magneten. 

% welche Probleme gab es mit NAO
Während der Messarbeiten sind einige Problematiken in der Arbeit mit dem NAO Roboter herausgetreten. Bereits bei der Programmierung gab es Grenzen. NAO ist ein Roboter, der für das Arbeiten mit Kindern und Jugendlichen konzipiert wurde und ist hauptsächlich ein Vorführungsobjekt. Die Gangarten sind begrenzt und ein Ausgleichssystem, welches dem Roboter eine Rückkopplung für Umgebungserkennung gewährleisten würde, lies sich nicht mit der Aufnahme von Messdaten vereinbaren. Dazu hätten externe Sensoren und weitere Geräte angeschlossen werden müssen. Hinzu kommt, dass NAOs Sensoren nicht genau genug sind, wie bereits in \cite{pressure_shoe} beschrieben. Das erschwert eine genaue Bestimmung von Stabilität und Aufwand. Des Weiteren konnte NAO zu Beginn bereits nicht geradeaus laufen. Dies musste manuell ausgeglichen werden und führte dazu, dass NAO u.U. nicht exakt dieselbe Strecke zurücklegte. Und schließlich begann der \texttt{moveTo()} Befehl per Zufall mit dem linken oder rechten Fuß zu erst. Dies hat zur Folge, dass die Aktoren unterschiedliche Ausgaben zu gleichen Zeiten haben und könnte die Mittelwerte verfälscht haben, welche zum Vergleich erstellt wurden. 

Da eine höhere Stabilität feststellbar war, könnte diese Art der Sohlenentwicklung durchaus interessant sein für künftige Konstruktionen in der Robotik. Außerdem ist die durch diese Arbeit entstandene Rampe vor allem für künftige Testläufe mit diversen Laufrobotern und Softrobotern geeignet. 
% ist es sinnvoll weiter an MAP Sohlen zu arbeiten. (Rampe erwähnen)

%%% Local Variables:
%%% mode: latex
%%% TeX-master: "main"
%%% End: