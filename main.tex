\documentclass[a4paper,
DIV=13,
12pt,
BCOR=10mm,
department=FakEI,
%lucida,
%KeepRoman,
oneside,
parskip=half,
automark,
%headsepline,
]{OTHRartcl}
\usepackage[ngerman]{babel}
\usepackage[T1]{fontenc}
\usepackage[utf8]{inputenc}
\usepackage{lmodern}

\usepackage{subcaption}
\usepackage{wrapfig,graphicx}
\usepackage{placeins}
\usepackage{amsmath}
\usepackage{amsfonts}
\usepackage{dsfont}
\usepackage{enumerate}
\usepackage{graphicx}
\usepackage{mathrsfs}
\usepackage{braket}
\usepackage{color}
\usepackage[colorlinks=true, linkcolor=cyan]{hyperref}
\usepackage[ugly]{units}

\usepackage{marvosym}
%\usepackage{hyperref}
%\hypersetup{
%	colorlinks = true,
%	linkcolor = cyan,
%	urlcolor = cyan
%}
% Default fixed font does not support bold face
\DeclareFixedFont{\ttb}{T1}{txtt}{bx}{n}{12} % for bold
\DeclareFixedFont{\ttm}{T1}{txtt}{m}{n}{12}  % for normal

\let\oldunit\unit
\renewcommand{\unit}[1]{\oldunit{\,#1}}
\renewcommand{\phi}{\varphi}
\newcommand{\diff}{\text{d}}
\renewcommand{\vec}[1]{\mathbf{#1}}

%\usepackage[singlelinecheck=off]{caption}
\setkomafont{captionlabel}{\bfseries}
\renewcommand*{\captionformat}{.~~}
\setcapindent{0pt}
\addto\captionsngerman{\renewcommand\figurename{Abb.}}


\usepackage{listings,xcolor}
\renewcommand{\lstlistingname}{Programmcode}
% Custom colors
\definecolor{pl_background}{rgb}{0.95,0.95,0.95}
\definecolor{pl_comment}{rgb}{0.12, 0.38, 0.18 }
\definecolor{pl_ifelse}{rgb}{0.74,0.74,.29}
\definecolor{pl_keyword}{rgb}{0.37, 0.08, 0.25}
\definecolor{pl_string}{rgb}{0.06, 0.10, 0.98}
\definecolor{deepblue}{rgb}{0,0,0.5}
\definecolor{deepred}{rgb}{0.6,0,0}
\definecolor{deepgreen}{rgb}{0,0.5,0}

% Vordefiniertes Programmlisting
\lstset{
	language = C,
	basicstyle = \small\sffamily,
	backgroundcolor = \color{pl_background},
	stringstyle = \color{pl_string},
	keywordstyle = \color{pl_keyword}\bfseries,
	commentstyle = \color{pl_comment}\itshape,
	frame = lrbt,
	numbers = left,
	showstringspaces = false,
	breaklines = true,
	xleftmargin = 15pt,
	emph = [1]{php},
	emphstyle = [1]\color{black},
	emph = [2]{if,and,or,else},
	emphstyle = [2]\color{pl_ifelse}
}

\lstset{
	language = Matlab,
	basicstyle = \small\sffamily,
	backgroundcolor = \color{pl_background},
	stringstyle = \color{pl_string},
	keywordstyle = \color{pl_keyword}\bfseries,
	commentstyle = \color{pl_comment}\itshape,
	frame = lrbt,
	numbers = left,
	showstringspaces = false,
	breaklines = true,
	xleftmargin = 15pt,
	emph = [1]{php},
	emphstyle = [1]\color{black},
	emph = [2]{if,and,or,else},
	emphstyle = [2]\color{pl_ifelse},
	tabsize=4
}

% Python style for highlighting
\newcommand{\pythonstyle}{\lstset{
		language=Python,
		basicstyle=\ttm,
		numbers = left,
		otherkeywords={self},             % Add keywords here
		keywordstyle=\ttb\color{deepblue},
		emph={MyClass,__init__},          % Custom highlighting
		emphstyle=\ttb\color{deepred},    % Custom highlighting style
		stringstyle=\color{deepgreen},
		backgroundcolor = \color{pl_background},
		breaklines = true,
		xleftmargin = \parindent,
		tabsize = 4,					% Aendert tab einzug, muss aber mit gobble abgestimmt werden
		gobble = 7,
		showstringspaces=false            % 
}}

\newcommand{\bashstyle}{\lstset{
		language = bash,
		basicstyle=\ttm,
		numbers = left,
		otherkeywords={self},             % Add keywords here
		keywordstyle=\ttb\color{deepblue},
		stringstyle=\color{deepgreen},
		backgroundcolor = \color{pl_background},
		breaklines = true,
		xleftmargin = \parindent,
		tabsize = 4,					% Aendert tab einzug, muss aber mit gobble abgestimmt werden
		gobble = 7,
		showstringspaces=false  					
}}

% Python environment
\lstnewenvironment{python}[1][]
{
	\renewcommand{\lstlistingname}{Code}
	\pythonstyle
	\lstset{#1}
}
{}
\lstnewenvironment{bash}[1][]
{
	\renewcommand{\lstlistingname}{Code}
	\bashstyle
	\lstset{#1}
}{}

\usepackage[backend=bibtex, style=numeric-comp, sorting=none]{biblatex}
\bibliography{bib}
%\title{Intelligentes Schuhwerk für den humanoiden NAO Roboter basierend auf Magnetostiction zur Verbesserung der Bodenhaftung}
\title{Entwicklung eines magneto-aktiven Schuhwerks für den humanoiden NAO Roboter zur Verbesserung der Gangstabilität}
\author{Tamara Szecsey}
\documenttype{Masterarbeit}

\studentid{3140789}
\department{Elektro- und Informationstechnik}
\studyprogramme{Master Electrical- and Microsystem Engineering}
\startingdate{1.\,Juni 2020}
\closingdate{31.\,März 2021}

\firstadvisor{Prof. Dr. Gareth Monkman}
\secondadvisor{Dr. Dirk Sindersberger}
%\externaladvisor{Dr. Klara Endlos}
\externallogo[height=1.6cm]{Bilder/mru.png}

\begin{document}
%	\numberwithin{equation}{section}
\maketitle
%\cleardoublepage
\makedeclaration

\thispagestyle{empty}
\tableofcontents	
\clearpage	
\setcounter{page}{1}

\section{Einleitung}
%* über humanoide Roboter (Vor- und Nachteile) \\
%* über nao und seine beschaffenheit \\
%* über map \\
%* Ziel der Arbeit \\
%* magnetostiction
%
%* Aufbau des Schuhs
%* 
Kapitel \ref{kap_MAP} erklärt die Definition und Eigenschaften von Magneto-aktiven Polymeren (MAP), welche als Sohle für den Nao Roboter eingesetzt wurden. Diese Erklärungen basieren auf einem Buch von Pelteret und Steinmann \cite{map2020}, welches ich für tiefergehende Lektüren empfehle. 
\newpage
\section{Theoretischer Hintergrund}

\begin{figure}[htb]
	\centering
	\includegraphics[width=0.7\linewidth]{Bilder/hardware_llegjoint.png}
	\caption{Position und mögliche Winkel der Aktoren des linken Beins.}
	\label{hardware_llegjoint}
\end{figure}
\begin{figure}[htb]
	\centering
	\includegraphics[width=0.7\linewidth]{Bilder/hardware_rlegjoint.png}
	\caption{Position und mögliche Winkel der Aktoren des rechten Beins.}
	\label{hardware_rlegjoint}
\end{figure}

%%% Local Variables:
%%% mode: latex
%%% TeX-master: "main"
%%% End:
% MAP
% Nao
% Software: Matlab, Autodesk Inventors, Naos Software
% Ziel der Arbeit
\newpage
\section{Versuchsaufbau}
In diesem Kapitel wird auf alle selbstkonstruierten Komponenten eingegangen, welche für die Messungen verwendet wurden. Zum einen gibt es die Schuhkonstruktion aus Abb. \ref{Schuh_Inventor}, die es erlaubt, die in einer Gussform angefertigten MAP Sohlen einzuhängen. Zum anderen wurde ein Laufsteg mit Rampenfunktion entworfen, welcher es erlaubt, Magnete unter die Lauffläche zu montieren. 

\subsection{Schuhkonstruktion} \label{Schuhkonstruktion}
\begin{figure}[tb]
	\hfill
	%\centering
	\begin{subfigure}[c]{.49\linewidth}
		\centering
		\includegraphics[width=\linewidth]{Bilder/Schuh_oben.png}
	\end{subfigure}
	\begin{subfigure}[c]{.49\linewidth}
		\centering
		\includegraphics[width=\linewidth]{Bilder/Schuh_unten.png}
	\end{subfigure}
	\hfill
	\caption{Linker Schuh in Autodesk Inventor, \textit{Links} von Oben, \textit{Rechts} von Unten.}
	\label{Schuh_Inventor}
	%\vspace{0.1cm}
\end{figure}
Die Hülle eines Fußes von NAO besteht aus einem zweiteiligen Oberteil, welches das Fußgelenk abschließt und einem unteren Teil, welcher mit vier Schrauben angebracht wird. Ohne Schraubenbefestigung liegt der Fuß nicht fest an. Die in Kapitel \ref{aufbau_NAO} beschriebenen vier Drucksensoren liegen dabei in dem unteren Teil auf Erhöhungen auf. Um die Auswirkungen anderer Sohlen für NAO messen zu können, muss der untere Teil des Fußes ausgetauscht werden. Dieser \glqq Schuh\grqq{} welcher die ursprüngliche Fußsohle ersetzt, enthält wiederum einen Steckplatz für die MAP Sohlen in verschiedenen Stärkegraden. 

Zur Formabnahme wurden sowohl die obere als auch die untere Hülle des Fußes von NAO 3D gescannt und in Polygonnetzformat (engl. \textit{mesh}) in Autodesk Inventor eingespeist. Somit konnten alle Maße direkt abgenommen und eingehalten werden. 

Im linken Bild von Abb. \ref{Schuh_Inventor} sind die vier flachen Zylinder zu sehen, auf denen die Drucksensoren aufliegen. Die vier Zylinder an den Seiten sind die Führung der Schrauben, welche an das obere Teil des Fußes von NAO geschraubt werden. Die Außenform umschließt den oberen Teil während eine zweite Erhöhung, welche an den Innenseiten verläuft, auf dem Rand des oberen Teils aufsitzt, sodass die Passform fest ineinander greift. Da das gesamte Gewicht des NAO auf den vier Drucksensoren lastet, kann Druckmaterial für die restliche Gesamtfläche bis auf stabilitätserhaltende Streben eingespart werden. Diese Form wurde mit dem Shape Generator von Autodesk Inventor generiert, sodass sie längs bis zu einem $45^\circ$ Winkel ohne zu brechen gebogen werden kann. Die instabilsten Stellen sind die Zylinder der Schraubvorrichtung, welche durch Fillets verstärkt wurden. Diese Instabilität ist auf den schichtweise Druckvorgang durch das FDM Verfahren geschuldet und kann durch kleinere Schichthöhen ausgeglichen werden.

Die Unterseite, zu sehen im rechten Bild von Abb. \ref{Schuh_Inventor}, ist ein Hohlraum für die MAP Sohle zusammen mit den viereckigen Steckeinlässen für die Halterung. Die Seiten des Schuhs sind so hoch, dass das MAP etwa $1 \unit{mm}$ herausragt. Andernfalls würde NAO auf der Schuhkante laufen und nicht auf dem MAP.  
% wesentliche Aufgabe des Schuhs
% Stabiler Ersatz der vorherigen Sohle -> hält nur mit den Schrauben
% Druckverteilung auf die Sensoren 
% Einsparung von Material in der Mitte
% Anpassung an den oberen Teil durch Auflage und Umfassung
% Einfassung des MAPs
% Basiert auf dem Scan des zweiteiligen Fußes von NAO.
% 4 Schrauben befestigen den unteren Teil an das obere Teil. Der Schuh ersetzt die eigentliche Sohle
% Das Gewicht des Roboters lastet hauptsächlich auf den 4 Drucksensoren, welche in Kap (Theorie) bereits erklärt wurden. Die 4 kreisförmigen Flächen liegen deshalb genau da an, wo diese Sensoren auch in dem ursprünglichen Teil anlagen. 
% Shape generater tragende Oberfläche.

\subsection{Herstellung des MAP} \label{Herstellung_MAP} \FloatBarrier
\begin{figure}[b!]
	\hfill
	%\centering
	\begin{subfigure}[c]{.49\linewidth}
		\centering
		\includegraphics[width=\linewidth]{Bilder/Gussform_Innenteil_verschoben.png}
	\end{subfigure}
	\begin{subfigure}[c]{.49\linewidth}
		\centering
		\includegraphics[width=\linewidth]{Bilder/Gussform.png}		
	\end{subfigure}
	\hfill
	\caption{Gussform der MAP Sohlen. \textit{Links} ist die Innenhalterung herausgenommen. \textit{Rechts} ist sie eingespannt in den sechs Eckhalterungen.}
	\label{Gussform_Inventor}
	%\vspace{0.1cm}
\end{figure}
Zunächst muss das Verhältnis für den Anteil des CIPs bestimmt werden. Die Masse ergibt sich aus
\begin{equation}
\text{m}_{\text{CIP}} = \frac{\text{Ratio}_{\text{CIP}} [\%]}{100\,\%}\cdot\text{m}_{ges}.
\end{equation}
Die beiden additiven Komponenten A und B, welche auch als Basis und Katalysator bezeichnet werden, sind im Verhältnis 1:1 zu mischen. Das Volumen in Milliliter ergibt sich aus:
\begin{align}
\text{V}_\text{A} &= \frac{\text{m}_\text{A}}{\rho_\text{A}} ,& 
\text{V}_\text{B} &= \frac{\text{m}_\text{B}}{\rho_\text{B}}
\end{align}
mit $\rho_\text{A} = 1,071 \unit{kg/ml}$ sowie $\rho_\text{B} = 1,046 \unit{kg/ml}$ sowie
\begin{align}	
\text{m}_\text{A} &= \left( 1- \frac{\text{Ratio}_{\text{CIP}} [\%]}{100\,\%}\right)\cdot
\frac{\alpha}{\alpha + \beta}\cdot \text{m}_{ges} ,&
\text{m}_\text{B} &= \left( 1- \frac{\text{Ratio}_{\text{CIP}} [\%]}{100\,\%}\right)\cdot
\frac{\beta}{\alpha + \beta}\cdot\text{m}_{ges},
\end{align}
Die Parameter $\alpha$ und $\beta$ stehen für das Mischverhältnis der jeweiligen Komponenten und sind in dieser Situation identisch 1, sodass die Formeln vereinfacht werden können:
\begin{align}	
\text{m}_\text{A} &= \left( 1- \frac{\text{Ratio}_{\text{CIP}} [\%]}{100\,\%}\right)\cdot
\frac{\text{m}_{ges}}{2} ,&
\text{m}_\text{B} &= \left( 1- \frac{\text{Ratio}_{\text{CIP}} [\%]}{100\,\%}\right)\cdot
\frac{\text{m}_{ges}}{2}.
\end{align}

Mit der Laborwaage ABT 120-5DM von Kern wird $\text{m}_\text{CIP}$ in einem Becher abgemessen. Mit den Spritzen lassen sich $\text{V}_\text{A}$ und $\text{V}_\text{B}$ mit einer Messgenauigkeit von einer Nachkommastelle beifügen. Es handelt sich um SF13 2k-Silikon vom Hersteller Silikon Fabrik. Der Becherinhalt wird dann mit einem kleinen Mixstab gemischt, um eine gleichmäßige Verteilung der beiden Komponenten zu erreichen und somit eine optimale Vernetzung zu gewährleisten. Anschließend wird die Probe in einen Exsikkator gestellt, welcher mit einer Vakuumpumpe evakuiert wird, um die Entgasung des Silikons zu erreichen. Schließlich kann das bis dahin noch flüssige MAP in die Gussform gegossen werden, nach spätestens einem Tag ist die Sohle dann komplett vernetzt. 

Da Silikon selbst sich nur sehr schlecht durch etwaige Klebstoffe nach der Vernetzung verkleben lässt, wird hier wie in Abb. \ref{Gussform_Inventor} zu sehen ist, eine $2\unit{mm}$ dicke Stangenkonstruktion eingehängt, welche bis auf die sechs Enden mit MAP umschlossen wird. Dieses aus PLA gedruckte Konstrukt ist flexibel und kann deshalb durch Verbiegen in die Verankerungen gedrückt werden. Nach der vollständigen Vernetzung kann die Sohle aus der Form entnommen und in den Schuh aus dem vorherigen Kapitel eingesetzt werden. 

Die vier Zylinder dienen als Platzhalter um die sechs Ecken in der Halterung des Schuhs für einen besseren Halt festzukleben und dann durch die Löcher des MAPs die Schrauben lockern zu können. 

Das Silikon selbst hat eine zu große Haftung, v.a. durch die Fläche des Schuhs. Deshalb wird es vor der Messung mit Speisestärke eingedeckt, was eine Bodenhaftung ähnlich der Plastiksohle des Originalschuhs von NAO zur Folge hat. Dies verhindert außerdem ungewollte Adhäsion.

\subsection{Laufstegkonstruktion} \FloatBarrier

Der NAO Roboter ist für den Einsatz auf geraden Bodenflächen im Innenbereich ausgelegt, wobei er bei einem Bewegungsablauf ohne Anpassung an die Umwelt wie mit dem Befehl \texttt{moveTo()} durch Rutschen nicht immer die gleiche Strecke zurücklegt. 
Um wiederholbare Messreihen garantieren zu können ist eine Teststrecke von Nöten. Des Weiteren sind verschiedene, flache Untergründe für eine Sohlenentwicklung interessant.
Außerdem kann auf das MAP nur Einfluss genommen werden, wenn ein magnetisches Feld angelegt wird. Deshalb wurde ein Laufsteg mit einem Hohlraum angefertigt, um unter der Fläche, auf der NAO läuft, Magneten anzubringen. 

\begin{wrapfigure}{hr}{0.4\linewidth}
	\vspace{-0.5cm}
	\centering
	\includegraphics[width=\linewidth]{Bilder/Rampe_Seitenansicht3.png}
	\caption{Seitenansicht der Rampe mit einer Breite von $66,4 \unit{cm}$.}
	\label{Rampe_Seite_Inventor}
	\vspace{-0.5cm}
\end{wrapfigure}

% Aufbau der Rampe
Der Laufsteg besteht aus einer $(120\times66,4) \unit{cm^2}$ großen Pressholzplatte, die auf der Oberseite mit einem Aluminiumkonstrukt erweitert ist, welches die Einschubplatten von beiden Längsseiten und nach oben hin abschließt, Abb. \ref{Rampe_Inventor} links. Auf den kurzen Seiten verriegeln jeweils zwei drehbare Keile den Einschub, sodass die Platten eingeschlossen werden, Abb. \ref{Rampe_Seite_Inventor}.

Auf der Unterseite sind an den Längsseiten zwei mit T-Nut versehene Aluminiumstangen angebracht, sowie eine zweiteilige Stangenkonstruktion, die eine Winkelverstellung mit Raste erlaubt, zu sehen in Abb. \ref{Rampe_Inventor} rechts. Die einstellbaren Winkel betragen ca. $5^\circ$ bis $17^\circ$, oder es wird für $0^\circ$ vollständig eingeklappt.
  
Es gab bereits ähnliche Aufbauten mit schräger Fläche, vgl. \cite{Lutz_naowalking}. Dies erfordert einen komplett anderen Gang und wäre über den zeitlichen Rahmen dieser Arbeit hinausgegangen. Die Neodymmagnete, die verwendet wurden, haben eine Haftkraft von ca. $16 \unit{kg}$, eine Maße von $(40\times40\times4) \unit{mm^3}$ \cite{schraubmagnet} und wurden an die Unterseite der Rampe geschraubt. Die ersten Versuche ergaben schließlich, dass das MAP nur bei einem Abstand ohne Einlageplatten reagierte. Deshalb wurden in den gesamten Messungen ohne diese Platten durchgeführt. 
\begin{figure}[tb]
	\hfill
	%\centering
	\begin{subfigure}[c]{.49\linewidth}
		\centering
		\includegraphics[width=\linewidth]{Bilder/Rampe_oben.png}
	\end{subfigure}
	\begin{subfigure}[c]{.49\linewidth}
		\centering
		\includegraphics[width=\linewidth]{Bilder/Rampe_unten.png}
	\end{subfigure}
	\hfill
	\caption{Laufstegrampenkonstruktion mit zwei austauschbaren Platten und einer Winkelverstellung mit Raste. \textit{Links:} Sicht von schräg oben mit eingeklappter Winkelverstellung. \textit{Rechts:} Sicht von schräg unten mit niedrigster Winkeleinstellung.}
	\label{Rampe_Inventor}
	%\vspace{0.1cm}
\end{figure}
\FloatBarrier
% Was man für die Messungen verwendet hat

%\subsubsection{Genaueres zum Einsatz an dem Nao Roboter}

%%% Local Variables:
%%% mode: latex
%%% TeX-master: "main"
%%% End:
% Schuh
% MAP Herstellung und Anbringung
% Rampe
%
\newpage
\section{Versuchsdurchführung}
Bereits in den vorherigen Kapiteln wurde auf die Durchführung dieses Versuchs eingegangen. Im folgenden wird über den Ablauf und einige Problematiken gesprochen, welche sich während dem Herstellungs- und Laufprozess herauskristallisierten. 

Wie bereits in Kapitel \ref{Herstellung_MAP} beschrieben, wird zunächst das MAP hergestellt und anschließend in die Formen gegossen. Während der Vermengung wurde festgestellt, dass bei 60\,\%gem MAP die Durchmischung nicht gewährleistet werden kann, solang der verwendete Becher über die Hälfte voll ist. Deshalb wurden in diesem Fall zwei identische Proben hergestellt, welche erst in der Form zusammengegossen wurden. Alle anderen Proben mussten nach der Durchmischung in zwei separate Becher umverteilt werden, da während der Entgasung das Gemisch an Volumen bis zu einer Druckabnahme von etwa $15 \unit{mbar}$ zunimmt. Dies hat zur Folge, dass der Becher nur etwa halb voll sein darf, damit das MAP nicht \glqq überkocht\grqq{}. 

Nach spätestens 24 Stunden ist das Silikon vollständig ausgehärtet und kann aus der Form gelöst werden. Aufgrund der Einhängevorrichtung musste die Gussform hierbei aufgebrochen werden, um die nur $2 \unit{mm}$ dicken Stangen nicht abzubrechen oder aus dem MAP zu reißen. Zudem verkeilt sich das Silikon in den Unebenheiten des 3D Drucks, sodass selbst mithilfe von Silikonöl, welches vor dem Gießen in die Form gegeben werden kann, sich das MAP nur schwer lösen lässt. 



% Probleme bei der Herstellung
% Tatsächlicher Messablauf
% Keine Messung der Schräglage

\begin{figure}[tb]
	\hfill
	\begin{subfigure}[c]{0.35\linewidth}
		\centering
		\includegraphics[width=\linewidth]{Bilder/Schuh_an_NAO_ohne_Sohle.jpg}
	\end{subfigure}
	\hfill
	\begin{subfigure}[c]{0.622\linewidth}
		\centering
		\includegraphics[width=\linewidth]{Bilder/Schuh_an_NAO_mit_Sohle.jpg}
	\end{subfigure}
	\hfill
	\caption{\textit{Links:} Der in Abb. \ref{Schuh_Inventor} in Autodesk Inventor erstellte Schuh aus PETG befestigt an der Unterseite des Fußes von NAO. \textit{Rechts:} Die aus der Gussform aus Abb. \ref{Gussform_Inventor} entnommene Sohle mit 20\,\% MAP Anteil befestigt in dem aus PETG gedruckten Schuh.}
	\label{nao_mit_schuhen}
\end{figure}

\begin{figure}[tb]
	\hfill
	\begin{subfigure}[c]{0.4\linewidth}
		\centering
		\includegraphics[width=\linewidth]{Bilder/NAO_auf_Rampe2.jpg}
	\end{subfigure}
	\hfill
	\begin{subfigure}[c]{0.4315\linewidth}
		\centering
		\includegraphics[width=\linewidth]{Bilder/magneten_an_rampe1_geschnitten.jpg}
	\end{subfigure}
	\hfill
	\caption{\textit{Links:} Der NAO Roboter steht im Ruhezustand an der Startposition auf der Rampe ohne Einlageplatten (an der Wand links neben NAO). \textit{Rechts:} Rampenunterseite mit befestigten Neodymmagneten.}
	\label{nao_und_rampe}
\end{figure}




\newpage
\section{Auswertung und Interpretation}
In diesem Kapitel werden die aufgenommenen Messungen ausgewertet und deren Bedeutung graphisch analysiert. Hierbei werden die Aufnahmen von zwei dimensionalen Ausgaben, welche das Gleichgewicht des Roboters widerspiegeln und die Werte der Aktoren in den Gelenken unterschieden.

\subsection{Gleichgewichtssensoren}

Wie in Kapitel \ref{software} beschrieben, werden zwei dimensionale Ausgaben, wie die x- und y-Ausgabe des Gyroskops, in einem Scatterhistogramm dargestellt, welches es ermöglicht, die Dichteverteilung einzusehen. 

\begin{figure}[b!]
	\centering
	\begin{adjustwidth}{-0.2\linewidth}{-0.2\linewidth}
		\hspace{+35pt}
		\begin{subfigure}[c]{.5\linewidth}
			\centering
			\includegraphics[width=\linewidth]{Bilder/Gyr_Grund_20_40_60_ohneM.pdf}
			\vspace{5pt}
		\end{subfigure}
		\hspace{-35pt}
		\begin{subfigure}[c]{.5\linewidth}
			\centering
			\includegraphics[width=\linewidth]{Bilder/Gyr_Grund_20_40_60_mitM.pdf}
			\vspace{5pt}
		\end{subfigure}
	\end{adjustwidth}
	\caption{Ausgabe des Gyroskops, x-Achse auf y-Achse, links ohne Magneten, rechts mit Magneten}\label{Gyr}
\end{figure}

In Abb. \ref{Gyr} sind die Aufnahmen des Gyroskops zu erkennen. Jede Ausgabe besteht aus 220 Punkten, da dies die Anzahl der Messdurchläufe pro Lauf sind. Jeder Wert wurde durch das arithmetische Mittel aus Gleichung \eqref{mean} aller Messpunkte zu diesem Zeitpunkt ermittelt. Der Graph auf der linken Seite ist die Ausgabe für die Messungen ohne Magneten an der Rampenunterseite, auf der rechten Seite die für die Messungen mit Magneten. Die jeweils zwei errechneten Dichteverteilung am linken und am unteren Rand des Hauptplots spiegeln die Verteilung des Scatterplots eindimensional wider. Die vereinzelten Punkte unterhalb von $-0,3 \unit{rad/s}$ sind dem Beginn der Messung geschuldet, während derer NAO sich in aufrechtem Stand befindet. Sobald der Roboter die laufende Haltung einnimmt, wird der Torso deshalb relativ schnell bewegt. Dies wird während der restlichen Aufnahmen nur passieren, wenn NAO umfällt. 
%Sollte ich hierfür das Gyroskop einmal auf die Zeit gesehen zeigen?

% abb 13: wie sind die Graphen aufgeteilt, was haben die Graphen an der seite zu bedeuten. Warum sind da vereinzelnd punkte in der mitte unten

Bei einem stabilen Gang würde man erwarten, dass die Geschwindigkeit, in der sich der Torso bewegt, gering bleibt. Dies bedeutet, je mehr Ausschwankungen zu sehen sind, desto instabiler läuft dieser Roboter und desto eher würde er das Gleichgewicht verlieren. 
% was würde man erwarten, wenn NAO stabil läuft

In Abb. \ref{Gyr} ist eine Tendenz von stabiler werdendem Gang von ohne Magneten zu mit Magneten zu erkennen. Allerdings fällt dies auch für den Originalschuh auf, welcher kein MAP enthält. Grund hierfür könnte der etwa 1cm breite Magnet sein, welcher den unteren Teil des Schuhs in Stellung hält, bevor die Schrauben zur Befestigung verwendet werden.

Des Weiteren heben sich die Sohlen mit $20\,\%$ CIP Anteil besonders hervor. Durch die Magneten erfahren sie ebenfalls eine minimale Verbesserung. Hierbei sollte erwähnt werden, dass diese Sohle mit der gedruckten Halterung dem Gewicht der Originalsohle am nächsten kommt und dadurch die Stabilität vielleicht besser gewährleistet ist. Die stärkste Auswirkung sowohl für die x- als auch die y-Achse gab es für $60\,\%$ CIP Anteil. Dies ist nicht verwunderlich, da hier die größte magnetische Kraft wirkt. 
% erst dann: man sieht eine Tendenz, dass es mit Magneten stabiler ist. 20% ist von vorn herein sehr stabil, der Grundschuh sowie 40% zeigen zeigt in y-Achse eine starke verbesserung mit Magneten, 60% zeigt in allen Achsen eine Verbesserung der Stabilität

\begin{figure}[htb]
	\centering
	\begin{adjustwidth}{-0.2\linewidth}{-0.2\linewidth}
		\hspace{+45pt}
		\begin{subfigure}[c]{.45\linewidth}
			\centering
			\includegraphics[width=\linewidth]{Bilder/Beschleunigung_Grund_20_40_60_ohneM.pdf}
			\vspace{5pt}
		\end{subfigure}
		%\hfill
		\hspace{-25pt}
		\begin{subfigure}[c]{.45\linewidth}
			\centering
			\includegraphics[width=\linewidth]{Bilder/Beschleunigung_Grund_20_40_60_mitM.pdf}
			\vspace{5pt}
		\end{subfigure}
	\end{adjustwidth}
	\caption{Ausgabe des Beschleunigungssensors, x-Achse auf y-Achse aufgetragen, links ohne Magneten, rechts mit Magneten} \label{Acc}
\end{figure}
Die Beschleunigungssensoren messen wie bereits erwähnt in $\unit{m/s^2}$ die Beschleunigung des Torsos. Deshalb ist ebenfalls zu erwarten, dass bei einem stabilen Gang diese Werte sich in der Nähe des Nullpunkts zentrieren, sollte NAO stabil laufen. Genau wie in Abb. \ref{Gyr} zeigen die Graphen in Abb. \ref{Acc} eine Tendenz zu einem stabileren Lauf mit Magneten. Auch hier wirken die $20\,\%$igen Proben außergewöhnlich stabil, während die $60\,\%$igen Proben in x-Richtung von den Magneten am meisten Profitieren. Gleichzeitig verhalten sich $40$ zu $60\,\%$ in y-Richtung ähnlich. 
% abb 14: Gleiche Graphenaufteilung, ähnliches Ergebnis

\begin{figure}[tb]
	\centering
	\begin{adjustwidth}{-0.2\linewidth}{-0.2\linewidth}
		\hspace{45pt}
		\begin{subfigure}[c]{.45\linewidth}
			\centering
			\includegraphics[width=\linewidth]{Bilder/Winkel_Grund_20_40_60_ohneM.pdf}
			\vspace{5pt}
		\end{subfigure}
		\hspace{-20pt}
		%\hfill
		\begin{subfigure}[c]{.45\linewidth}
			\centering
			\includegraphics[width=\linewidth]{Bilder/Winkel_Grund_20_40_60_mitM.pdf}
			\vspace{5pt}
		\end{subfigure}
	\end{adjustwidth}
	\caption{Echtzeit errechneter Winkel, x-Achse auf y-Achse aufgetragen, links ohne Magneten, rechts mit Magneten} \label{Angle}
\end{figure}
Die Winkelwerte in Abb. \ref{Angle} entstehen aus der Berechnung durch Gyroskop und Beschleunigungssensor. Da beide alle Graphen zuvor eine Tendenz von höherer Stabilität durch Magneten aufwiesen, ist zu erwarten, dass dies bei den Winkelgraphen auch der Fall ist. In der y-Achse ist eine Verbesserung von $40$ und $60\,\%$ erkennbar.In x-Richtung scheint $40\,\%$iges MAP am meisten zu profitieren, allerdings gilt für den Originalschuh eher das Gegenteil. 
% abb 15: Ist aus den Werten, welche in abb 13 und 14 zu sehen sind errechnet. deshalb auch hier ähnliches Ergebnis

\begin{figure}[tb]
	\centering
	\begin{adjustwidth}{-0.2\linewidth}{-0.2\linewidth}
		\hspace{40pt}
		\begin{subfigure}[c]{.45\linewidth}
			\centering
			\includegraphics[width=\linewidth]{Bilder/links_CoM_ohneM.pdf}
			\vspace{5pt}
		\end{subfigure}
		\hspace{-10pt}
		%\hfill
		\begin{subfigure}[c]{.45\linewidth}
			\centering
			\includegraphics[width=\linewidth]{Bilder/links_CoM_mitM.pdf}
			\vspace{5pt}
		\end{subfigure}
	\end{adjustwidth}
	\caption{Linker errechneter Massenschwerpunkt aufgenommen durch die FSR, x-Achse auf y-Achse aufgetragen, links ohne Magneten, rechts mit Magneten} \label{CoM_links}
\end{figure}
\begin{figure}[tb]
	\centering
	\begin{adjustwidth}{-0.2\linewidth}{-0.2\linewidth}
		\hspace{40pt}
		\begin{subfigure}[c]{.45\linewidth}
			\centering
			\includegraphics[width=\linewidth]{Bilder/rechts_CoM_ohneM.pdf}
			\vspace{5pt}
		\end{subfigure}
		\hspace{-10pt}
		%\hfill
		\begin{subfigure}[c]{.45\linewidth}
			\centering
			\includegraphics[width=\linewidth]{Bilder/rechts_CoM_mitM.pdf}
			\vspace{5pt}
		\end{subfigure}
	\end{adjustwidth}
	\caption{Rechter errechneter Massenschwerpunkt aufgenommen durch die FSR, x-Achse auf y-Achse aufgetragen, links ohne Magneten, rechts mit Magneten} \label{CoM_rechts}
\end{figure}
Das hier verwendete Modell von NAO bietet noch die Möglichkeit die Stabilität des Gangs über die Drucksensoren in den Fußsohlen, auf die in Kapitel \ref{aufbau_NAO} bereits eingegangen wurde, zu messen. Die errechneten Werte der zweidimensionalen Massenschwerpunkte durch die 8 FSR Sensoren sind in Abb \ref{CoM_links} und \ref{CoM_rechts} zu sehen.
% abb 16 und 17 sind aus den FSR Daten entstanden und weißen eine ebenfalls weisen nur für die MAP sohlen unterschiede auf, aber kaum für den Grundschuh. 

% Warum diese Auswertung mit Vorsicht zu genießen ist: frühere Auswertung
% Ungenauigkeiten der FSR
% Dennoch wird bei allen Graphen eine verbesserung mit Magneten festgestellt, warum könnte das so sein?


%Notizen:
%
%Die folgenden Graphen unterscheiden sich in ihrer Darstellung: Zum einen wurden Gyroskop, Beschleunigungssensors, Winkel und Center of Mass in sog. Scatterhistogrammen geplottet, siehe Abb \ref{Gyr},\ref{Acc},\ref{Angle}. Der Hauptplot ist ein scatter-Plot, welcher die Verteilung auf der X verglichen zur Y-Achse zeigt. An den Rändern sind jeweils die Histogramme der Achsen aufgetragen, welche die Wahrscheinlichkeitsverteilung angeben. Dabei ist die Annahme, dass ein stabilierer Gang eine steilere Wahrscheinlichkeitsverteilung hervorbringt, da sich die Aufnahmepunkte in der Mitte häufen müssten. Eine flache Verteilungskurve würde im Gegenzug bedeuten, dass NAO während dem Gang große Schwankungen aufweist und deshalb instabiler läuft.
%
%Es lässt sich eine Tendenz erkennen, dass der Gang mit Magneten stabiler ist, als ohne. Außerdem zeichnet sich das 20\,\% MAP durch besonders gute Stabilität hab. 
%
%Der Strom wurde in Histogrammen aufgetragen, welche die relative Häufigkeit des Stroms während der Messung aufzeigt. Die Graphen \ref{AnklePitch_Current_links},\ref{AnklePitch_Current_rechts},\ref{AnkleRoll_Current_links},\ref{AnkleRoll_Current_rechts} sind unterteilt in zwei Arten von Plots. Einerseits ist ein Histogramm mit jeweils 20 Säulen zu sehen. Diese wurde mit der Wahrscheinlichkeitsdichtefunktion von Matlab (pdf) ergänzt, um die Verteilung besser einsehen zu können.
%
%Die Annahme ist, dass ein höherer Strom mehr Reibungs- oder Haftwiderstand bedeuten. Leider ist die Datenlage hier sehr uneindeutig. 

\FloatBarrier
\subsection{Sensoren der Aktoren}

\begin{figure}[tb]
	\centering
%	\begin{adjustwidth}{-0.2\linewidth}{-0.2\linewidth}
%		\hspace{5pt}
		\begin{subfigure}[c]{.9\linewidth}
			\centering
			\includegraphics[width=\linewidth]{Bilder/links_Current_AnklePitch_ohneM.pdf}
%			\vspace{5pt}
		\end{subfigure}
%		\hspace{20pt}
%		\hfill
		\begin{subfigure}[c]{.9\linewidth}
			\centering
			\includegraphics[width=\linewidth]{Bilder/links_Current_AnklePitch_mitM.pdf}
%			\vspace{5pt}
		\end{subfigure}
%	\end{adjustwidth}
	\caption{AnklePitch gemessener Strom im linken Fuß, Strom in Ampère aufgetragen auf die Häufigkeit. Das Histogramm wurde ergänzt durch eine Wahrscheinlichkeitsdichtefunktion. Der obere Graph sind die Aufnahmen ohne Magneten, der untere mit Magneten.} \label{AnklePitch_Current_links}
\end{figure}
\begin{figure}[tb]
	\centering
%	\begin{adjustwidth}{-0.2\linewidth}{-0.2\linewidth}
%		\hspace{5pt}
		\begin{subfigure}[c]{.9\linewidth}
			\centering
			\includegraphics[width=\linewidth]{Bilder/rechts_Current_AnklePitch_ohneM.pdf}
			\vspace{5pt}
		\end{subfigure}
%		\hspace{20pt}
		\hfill
		\begin{subfigure}[c]{.9\linewidth}
			\centering
			\includegraphics[width=\linewidth]{Bilder/rechts_Current_AnklePitch_mitM.pdf}
			\vspace{5pt}
		\end{subfigure}
%	\end{adjustwidth}
	\caption{AnklePitch gemessener Strom im rechter Fuß, Strom in Ampère aufgetragen auf die Häufigkeit. Das Histogramm wurde ergänzt durch eine Wahrscheinlichkeitsdichtefunktion. Der obere Graph sind die Aufnahmen ohne Magneten, der untere mit Magneten.} \label{AnklePitch_Current_rechts}
\end{figure}

\begin{figure}[tb]
	\centering
	%	\begin{adjustwidth}{-0.2\linewidth}{-0.2\linewidth}
	%		\hspace{5pt}
	\begin{subfigure}[c]{.9\linewidth}
		\centering
		\includegraphics[width=\linewidth]{Bilder/links_Current_AnkleRoll_ohneM.pdf}
		%			\vspace{5pt}
	\end{subfigure}
	%		\hspace{20pt}
	%		\hfill
	\begin{subfigure}[c]{.9\linewidth}
		\centering
		\includegraphics[width=\linewidth]{Bilder/links_Current_AnkleRoll_mitM.pdf}
		%			\vspace{5pt}
	\end{subfigure}
	%	\end{adjustwidth}
	\caption{AnkleRoll gemessener Strom im linken Fuß, Strom in Ampère aufgetragen auf die Häufigkeit. Das Histogramm wurde ergänzt durch eine Wahrscheinlichkeitsdichtefunktion. Der obere Graph sind die Aufnahmen ohne Magneten, der untere mit Magneten.} \label{AnkleRoll_Current_links}
\end{figure}
\begin{figure}[tb]
	\centering
	%	\begin{adjustwidth}{-0.2\linewidth}{-0.2\linewidth}
	%		\hspace{5pt}
	\begin{subfigure}[c]{.9\linewidth}
		\centering
		\includegraphics[width=\linewidth]{Bilder/rechts_Current_AnkleRoll_ohneM.pdf}
		\vspace{5pt}
	\end{subfigure}
	%		\hspace{20pt}
	\hfill
	\begin{subfigure}[c]{.9\linewidth}
		\centering
		\includegraphics[width=\linewidth]{Bilder/rechts_Current_AnkleRoll_mitM.pdf}
		\vspace{5pt}
	\end{subfigure}
	%	\end{adjustwidth}
	\caption{AnkleRoll gemessener Strom im rechter Fuß, Strom in Ampère aufgetragen auf die Häufigkeit. Das Histogramm wurde ergänzt durch eine Wahrscheinlichkeitsdichtefunktion. Der obere Graph sind die Aufnahmen ohne Magneten, der untere mit Magneten.} \label{AnkleRoll_Current_rechts}
\end{figure}

%\subsection{Messungen von AnkleRoll zum Test}
%Die Messungen von LAnkleRoll und RAnkleRoll, zu sehen in Abb. \ref{hardware_llegjoint} und \ref{hardware_rlegjoint} wurden 20 mal wiederholt und mit den normalen Schuhen von Nao vollzogen. Dabei legte er in etwa eine Strecke von $0,8 \unit{m}$ auf der Rampe im flachen Zustand zurück. Insgesamt wurden alle verfügbaren Messwerte von AnkleRoll aufgezeichnet, das sind pro Aktor 6 Messwerte. Temperatur, Stiffness und Temperatur Status erwiesen sich als Konstant und daher nicht entscheidend, um einen Unterschied der Bodenbeschaffenheit oder Sohlen erkennen zu können. Stiffness ist immer auf $100\%$ während dem Gang. 
%Der Befehl für diesen Lauf war der moveTo() Befehl, welcher nicht weiter verändert wurde (kommt in den Theorieteil).
%
%
%In Abb. \ref{AnkleRoll_links_act} und \ref{AnkleRoll_rechts_act} sind die Messdaten von jeweils einem Fuß des Messwertes Position/Actuator abgebildet. Hier ist zu sehen, dass die Anfangswerte sich aufspalten, in positive und einmal in negative Winkelangaben. Dies ist der Tatsache geschuldet, dass die Funktion moveTo() per Zufall Nao mit dem linken oder mit dem rechten Fuß beginnen lässt. 
%
%Dies wurde für Abb. \ref{AnkleRoll_beide_act_sens_links_anfang} und \ref{AnkleRoll_beide_act_sens_rechts_anfang} sortiert. In ersterer Abbildung beginnt Nao mit dem linken Fuß. Da die Hüfte sich für den ersten Schritt nach rechts bewegen muss, verschiebt sich die Position beider Gelenke in die Negativrichtung, der Winkel wird absolut gemessen, wie in Abb \ref{hardware_llegjoint} und \ref{hardware_rlegjoint} zu sehen ist. 
%
%Außerdem sind in Abb. \ref{AnkleRoll_beide_act_sens_links_anfang} und \ref{AnkleRoll_beide_act_sens_rechts_anfang} neben den Messwerten von Position Actuator in schwarz auch die von Position Sensor in blau gezeigt. \textcolor{red}{Was diese beiden Messwerte genau unterscheidet und ob einer von moveTo() vorgegeben wird, ist noch zu entscheiden.} Der bedeutenste Unterschied ist zu Beginn der Aufnahmen. Die Position/Actuator Messung beginnt nahe 0, während Position/Sensor für den jeweiligen Fuß bei einem Wert über Null oder unter Null anfängt. 
%
%Es ist eindeutig zu erkennen, dass die Messungen erst nach der Sortierung des Anfangsschrittes ein regelmäßiges Bild ergeben. 
%
%Der Strom, welcher die Gelenke einsetzen müssen um das gewollte Ergebnis zu erzielen, scheint eine mögliche, vergleichbare Aufnahmegröße für unterschiedliche Sohlen und Umgebungen des Nao zu sein. In Abb. \ref{AnkleRoll_beide_current_links_anfang} und \ref{AnkleRoll_beide_act_sens_rechts_anfang} ist der Messwert Current aufgeteilt in Anfangsschritte gezeigt. Hier ist der Unterschied, mit welchem Fuß der erste Schritt gemacht wird, nicht so gravierend, wie bei den vorherigen Messwerten. Allerdings zeichnet sich eine Tendenz ab, dass der linke Fuß, hier in schwarz, einen höheren Strom beansprucht, als der rechte Fuß. Dies könnte dem beobachteten Fehlgang des Naos und dem zusätzlichen Geräusch bei jedem zweiten Schritt geschultet sein. Bei normaler Einstellung und ohne Korrektur würde dieser Nao einen Bogen nach rechts laufen. Um dies auszugleichen wurden bei moveTo() Anpassungen hinzugefügt.   
%
%\begin{figure}[tb]
%	\centering
%	\includegraphics[width=1\linewidth]{Bilder/AnkleRoll_links_act.pdf}
%	\caption{AnkleRoll Messwert Position Actuator des linken Fußes}
%	\label{AnkleRoll_links_act}
%\end{figure}
%
%\begin{figure}[tb]
%	\centering
%	\includegraphics[width=1\linewidth]{Bilder/AnkleRoll_rechts_act.pdf}
%	\caption{AnkleRoll Messwert Position Actuator des rechten Fußes}
%	\label{AnkleRoll_rechts_act}
%\end{figure}
%\begin{figure}[tb]
%	\centering
%	\includegraphics[width=1\linewidth]{Bilder/AnkleRoll_beide_act_sens_links_anfang.pdf}
%	\caption{AnkleRoll Aktoren beider Seiten mit dem Position/Actuator Messwert in schwarz und dem Position/Sensor Messwert in blau. Nao macht hier den ersten Schritt mit Links.}
%	\label{AnkleRoll_beide_act_sens_links_anfang}
%\end{figure}
%\begin{figure}[tb]
%	\centering
%	\includegraphics[width=1\linewidth]{Bilder/AnkleRoll_beide_act_sens_rechts_anfang.pdf}
%	\caption{AnkleRoll Aktoren beider Seiten mit dem Position/Actuator Messwert in schwarz und dem Position/Sensor Messwert in blau. Nao macht hier den ersten Schritt mit Rechts.}
%	\label{AnkleRoll_beide_act_sens_rechts_anfang}
%\end{figure}
%\begin{figure}[tb]
%	\centering
%	\includegraphics[width=1\linewidth]{Bilder/AnkleRoll_beide_current_links_anfang.pdf}
%	\caption{AnkleRoll Aktoren beider Seiten mit dem Current Messwert. Messwert Links ist in Schwarz, Messwert Rechts ist in Blau. Nao macht hier den ersten Schritt mit Links.}
%	\label{AnkleRoll_beide_current_links_anfang}
%\end{figure}
%\begin{figure}[tb]
%	\centering
%	\includegraphics[width=1\linewidth]{Bilder/AnkleRoll_beide_current_rechts_anfang.pdf}
%	\caption{AnkleRoll Aktoren beider Seiten mit dem Current Messwert. Messwert Links ist in Schwarz, Messwert Rechts ist in Blau. Nao macht hier den ersten Schritt mit Rechts.}
%	\label{AnkleRoll_beide_current_rechts_anfang}
%\end{figure}

%%% Local Variables:
%%% mode: latex
%%% TeX-master: "main"
%%% End:

% Allgemeines zu diesem Nao (Vormessungen), Probleme der Messungenauigkeit
% Vergleichen der Messungen mit und ohne Magneten
%
%\newpage
%\section{Fazit und Ausblick}
%% zusammenfassen, was untersucht wurde
Diese Arbeit beschäftigte sich mit dem Konstruieren eines Schuhs für den von Softrobotics hergestellen humanoiden Roboter NAO. Ziel war es, das bisher bereits vielseitig angewendete, aber noch nicht an Roboterfüßen verwendete MAP zu testen, eine Teststrecke mit der Möglichkeit Magneten anzubringen und verschiedene Winkel einzustellen zu bauen und die Stabilität des NAO zu erhöhen. Des Weiteren sollten die Auswirkungen der neuen Lauffläche auf den Roboter getestet werden. 

% was waren die Ergebnisse
Zusammenfassend lässt sich, wie in Kapitel \ref{gleichgewicht} beschrieben, einen Zusammenhang zwischen erhöhter Stabilität und Magneten erkennen. Die Prozentanteile die verwendet wurden, unterscheiden sich sehr im Anteil von CIP und Gewicht. Die Reaktion auf das Magnetfeld waren für höhere Prozente ersichtlich, während der Originalschuh kaum Reaktion zeigte. Der Gang mit $20\,\%$ war ungewöhnlich stabil, sowohl mit als auch ohne Magneten. 

% welche Probleme gab es mit NAO
Während der Messarbeiten sind einige Problematiken in der Arbeit mit dem NAO Roboter herausgetreten. Bereits bei der Programmierung gab es Grenzen. NAO ist ein Roboter, der für das Arbeiten mit Kindern und Jugendlichen konzipiert wurde und ist hauptsächlich ein Vorführungsobjekt. Die Gangarten sind begrenzt und ein Ausgleichssystem, welches dem Roboter eine Rückkopplung für Umgebungserkennung gewährleisten würde, lies sich nicht mit der Aufnahme von Messdaten vereinbaren. Dazu hätten externe Sensoren und weitere Geräte angeschlossen werden müssen. Hinzu kommt, dass NAOs Sensoren nicht genau genug sind, wie bereits in \cite{pressure_shoe} beschrieben. Das erschwert eine genaue Bestimmung von Stabilität und Aufwand. Des Weiteren konnte NAO zu Beginn bereits nicht geradeaus laufen. Dies musste manuell ausgeglichen werden und führte dazu, dass NAO u.U. nicht exakt dieselbe Strecke zurücklegte. Und schließlich begann der \texttt{moveTo()} Befehl per Zufall mit dem linken oder rechten Fuß zu erst. Dies hat zur Folge, dass die Aktoren unterschiedliche Ausgaben zu gleichen Zeiten haben und könnte die Mittelwerte verfälscht haben, welche zum Vergleich erstellt wurden. 

Da eine höhere Stabilität feststellbar war, könnte diese Art der Sohlenentwicklung durchaus interessant sein für künftige Konstruktionen in der Robotik. Außerdem ist die durch diese Arbeit entstandene Rampe vor allem für künftige Testläufe mit diversen Laufrobotern und Softrobotern geeignet. 
% ist es sinnvoll weiter an MAP Sohlen zu arbeiten. (Rampe erwähnen)

%%% Local Variables:
%%% mode: latex
%%% TeX-master: "main"
%%% End:
%
\FloatBarrier
\newpage
\section{Anhang} \label{Anhang}
	\begin{python} [caption={Pythonprogramm für Messaufnahmen}, label=Messungscode]
		# ! /usr/bin/env python
		# -*- encoding: UTF-8 -*-
		
		"""Example: Use getData Method to Use FSR Sensors"""
		
		import qi
		import argparse
		import sys
		import time
		import csv
		import re
		import shutil
		from tempfile import mkstemp
		import os
		
		def sed(pattern, replace, source, dest=None, count=0):
			"""Reads a source file and writes the destination file.
			
			In each line, replaces pattern with replace.
			
			Args:
			pattern (str): pattern to match (can be re.pattern)
			replace (str): replacement str
			source  (str): input filename
			count (int): number of occurrences to replace
			dest (str):   destination filename, if not given, source will be over written.
			"""
			
			fin = open(source, 'r')
			num_replaced = count
			
			if dest:
				fout = open(dest, 'w')
			else:
				fd, name = mkstemp()
				fout = open(name, 'w')
			
			for line in fin:
				out = re.sub(pattern, replace, line)
				fout.write(out)
				
				if out != line:
				num_replaced += 1
				if count and num_replaced > count:
				break
			try:
				fout.writelines(fin.readlines())
			except Exception as E:
				raise E
			
			fin.close()
			fout.close()
			
			if not dest:
				shutil.move(name, source)
		
		
		def zeilen_aufteilen(file):
			sed(',platzhalter,', '\n', file)
			sed(',platzhalter', '', file)
		
		
		def recordData(memory_service):
			""" Get pressure sensor data from ALMemory
			Returns a matrix of values
			
			"""
			print "Recording data..."
			data = list()
			for range_counter in range(1, 230):
				#Gyroscope
				GyrX = memory_service.getData("Device/SubDeviceList/InertialSensor/GyroscopeX/Sensor/Value")
				GyrY = memory_service.getData("Device/SubDeviceList/InertialSensor/GyroscopeY/Sensor/Value")
				data.append(GyrX)
				data.append(GyrY)
				
				# Adding the summary of the FSR
				LFsrTw = memory_service.getData("Device/SubDeviceList/LFoot/FSR/TotalWeight/Sensor/Value")
				RFsrTw = memory_service.getData("Device/SubDeviceList/RFoot/FSR/TotalWeight/Sensor/Value")
				
				LFcopX = memory_service.getData("Device/SubDeviceList/LFoot/FSR/CenterOfPressure/X/Sensor/Value")
				LFcopY = memory_service.getData("Device/SubDeviceList/LFoot/FSR/CenterOfPressure/Y/Sensor/Value")
				RFcopX = memory_service.getData("Device/SubDeviceList/RFoot/FSR/CenterOfPressure/X/Sensor/Value")
				RFcopY = memory_service.getData("Device/SubDeviceList/RFoot/FSR/CenterOfPressure/Y/Sensor/Value")
				data.append(LFsrTw)
				data.append(RFsrTw)
				data.append(LFcopX)
				data.append(LFcopY)
				data.append(RFcopX)
				data.append(RFcopY)
				
				# LeftAnkleRoll
				PosAct = memory_service.getData("Device/SubDeviceList/LAnkleRoll/Position/Actuator/Value")
				PosSens = memory_service.getData("Device/SubDeviceList/LAnkleRoll/Position/Sensor/Value")
				ElectrSens = memory_service.getData("Device/SubDeviceList/LAnkleRoll/ElectricCurrent/Sensor/Value")
				data.append(PosAct)
				data.append(PosSens)
				data.append(ElectrSens)
				
				# RightAnkleRoll
				PosAct = memory_service.getData("Device/SubDeviceList/RAnkleRoll/Position/Actuator/Value")
				PosSens = memory_service.getData("Device/SubDeviceList/RAnkleRoll/Position/Sensor/Value")
				ElectrSens = memory_service.getData("Device/SubDeviceList/RAnkleRoll/ElectricCurrent/Sensor/Value")
				data.append(PosAct)
				data.append(PosSens)
				data.append(ElectrSens)
				
				# LeftAnklePitch
				PosAct = memory_service.getData("Device/SubDeviceList/LAnklePitch/Position/Actuator/Value")
				PosSens = memory_service.getData("Device/SubDeviceList/LAnklePitch/Position/Sensor/Value")
				ElectrSens = memory_service.getData("Device/SubDeviceList/LAnklePitch/ElectricCurrent/Sensor/Value")
				data.append(PosAct)
				data.append(PosSens)
				data.append(ElectrSens)
				
				# RightAnklePitch
				PosAct = memory_service.getData("Device/SubDeviceList/RAnklePitch/Position/Actuator/Value")
				PosSens = memory_service.getData("Device/SubDeviceList/RAnklePitch/Position/Sensor/Value")
				ElectrSens = memory_service.getData("Device/SubDeviceList/RAnklePitch/ElectricCurrent/Sensor/Value")
				data.append(PosAct)
				data.append(PosSens)
				data.append(ElectrSens)
				
				data.append('platzhalter')
				time.sleep(0.05)
			return data
		
		
		def count_files():
			counter = 1
			# str.zfill schreibt vor, wie lang die Zahl mit Nullen davor sein soll. also zfill(3) ist 3 Zahlen lang.
			filename = 'measurement' + str(counter).zfill(3) + '.csv'
			
			# Wenn das file nicht exisiert, erstelle measurement001.csv
			while os.path.exists(filename):
				counter = counter + 1
				filename = 'measurement' + str(counter).zfill(3) + '.csv'
			create_file(filename)
			return filename
		
		
		def create_file(filename):
			with open(filename, "w") as f:
				pass
		
		
		def main(session):
			"""
			This example uses the getData method to use FSR sensors.
			"""
			# Get the ALProxy ALMemory and ALMotion
			from naoqi import ALProxy
			memory_service = session.service("ALMemory")
			motion = ALProxy("ALMotion", "nao.local", 9559)
			
			# wake up nao
			motion.wakeUp()
			
			motion.moveInit()
			motion.post.moveTo(0.85, -0.10, -0.25, [["MaxStepFrequency", 0.0]])
			
			data = recordData(memory_service)
			filename = count_files()
			
			output = os.path.abspath(filename)
			with open(output, "wb") as file:
				writer = csv.writer(file, delimiter=',')
				writer.writerow(data)
			zeilen_aufteilen(output)
			print "Results written to", output
			# go back to crouch position and sleep
			motion.rest()
			
		
		if __name__ == "__main__":
			parser = argparse.ArgumentParser()
			parser.add_argument("--ip", type=str, default="127.0.0.1",
			help="Robot IP address. On robot or Local Naoqi: use '127.0.0.1'.")
			parser.add_argument("--port", type=int, default=9559,
			help="Naoqi port number")
			
			args = parser.parse_args()
			session = qi.Session()
			try:
				session.connect("tcp://" + args.ip + ":" + str(args.port))
			except RuntimeError:
				print ("Can't connect to Naoqi at ip \"" + args.ip + "\" on port " + str(args.port) + ".\n"
				"Please check your script arguments. Run with -h option for help.")
			sys.exit(1)
			main(session)	
	\end{python}
Der Programmcode \ref{Messungscode} kann in mehrere Funktionen aufgeteilt betrachtet werden. Die Funktion \texttt{sed} ist aus \cite{sed_python} entnommen und funktioniert wie die gleichnamige Funktion der Linux-Bash. Sie wird benötigt, um nach jedem Durchgang der Messschleife in \texttt{recordData} eine neue Zeile in die CSV Datei zu schreiben. Dies geschieht durch die Funktion \texttt{zeilen\_aufteilen}. \texttt{recordData} wurde aus den Beispielen der NAO Dokumentation \cite[/Other tutorials/Python SDK - Tutorials/Python SDK - Examples/Sensors]{naoqi_dev_guide} entnommen und angepasst, sodass am Ende jeder Zeile von \texttt{data} ein Platzhalter eingefügt wird und alle gewünschten Sensoren abgegriffen werden. Die Funktion \texttt{count\_files} sorgt dafür, dass keine vorhandenen Messungen überschrieben werden und jede Messdatei eine fortlaufende Nummerierung erhält. 

In der \texttt{main} Funktion werden ALMemory und ALMotion geladen, und der Gang einschließlich des Abgreifens der Sensorwerte ausgeführt. Die Ausgabe der Messwerte während dem Gang ist nur möglich durch den Präfix \texttt{post} vor \texttt{moveTo}.

Abschließend dient die letzte \texttt{if}-Abfrage zur Verbindung mit NAO, allerdings nur, wenn dieses Pythonprogramm selbst auf dem NAO liegt. Die Methode \texttt{post} sowie die Aufnahme der Sensoren während dem Lauf der Methode \texttt{moveTo} funktionieren nur lokal, deshalb ist es in diesem Fall nicht möglich, das Programm von dem eigenen Rechner aus zu starten. Mit anderen Methoden wäre eine Programmaufrufung über eine Wlan Verbindung durchaus möglich. Um Programme direkt auf dem NAO zu starten, wird eine \texttt{ssh}-Verbindung hergestellt und darüber dann \texttt{python} ausgeführt.
		
%%% Local Variables:
%%% mode: latex
%%% TeX-master: "main"
%%% End:

\FloatBarrier
\newpage
\clearpage	
%\thispagestyle{empty}
\printbibliography

\end{document}

%%% Local Variables:
%%% mode: latex
%%% TeX-master: "main"
%%% End: