% wer oder was ist NAO 
% Stabilität bei zweibeinern ist generell ein problem
% MAP was ist das und bisher hat das noch niemand gemacht.
In der Robotik werden seit jeher Roboter gebaut, die dem Menschen ähnlich sehen, und damit auf zwei Beinen laufen. Diese Gangart ist jedoch deutlich instabiler, als insektenähnliche Roboter oder Konstrukte mit vier oder mehr Rädern. Dennoch ist es durchaus sinnvoll humanoide Roboter zu entwickeln und verbessern, da sie erstens mit der für Menschen gemachten Umwelt besser zurecht kommen und zum anderen eine verbesserte, menschliche Interaktion ermöglichen. 

Magneto-aktive Polymere (MAP) sind Stoffe, die auf magnetische Felder reagieren und diese wurden in der Robotik zumeist in Greifern oder als Teilkomponente von Softrobotern verwendet. Das Material kam bisher auf der Lauffläche bei zweibeinigen Robotern nicht zum Einsatz und könnte flexiblere Oberflächenanpassung ermöglichen. 

Diese Arbeit handelt von der Konstruktion eines Schuhs, welcher an die Stelle der Schuhunterseite\hyphenation{Schuh-unter-sei-te} des humanoiden NAO Roboters angebracht wird und eine Sohle aus MAP hält. Ziel ist es hierbei, eine erhöhte Stabilität des zweibeinigen Roboters durch das Anlegen eines magnetischen Feldes bei einer Gangart ohne Rückkopplungsschleife, d.h. ohne Anpassung an die Umwelt durch Sensorenaufnahmen und autonomer Korrektur, zu erreichen. 

Durch die Analyse der Gleichgewichtssensoren des NAO wird gezeigt, dass durch den Einsatz von Magneten unter dem Laufgrund der Roboter stabiler läuft, wenn MAP Sohlen eingesetzt werden. Hierfür werden drei verschiedene Sohlen hergestellt und mit dem Originalschuh verglichen. Die Sohle, welche gewichtsmäßig dem Originalschuh am nächsten kommt, wird sich gleichzeitig als diejenige herausstellen, welche den stabilsten Gang des Roboters hervorbringt. 

Zunächst werden in Kapitel \ref{theorie} auf die Grundlagen der vorhandenen Materialien und Software eingegangen. Dies beginnt mit den Aufbau des NAO Roboter und geht dann über in die verwendete Software, bei der sowohl die Robotersoftware als auch die für die Erstellung der Konstruktionen im Folgekapitel und für die Auswertung nötigen Programme vorgestellt werden. Schließlich wird die Definition und Eigenschaften von magneto-aktiven Polymeren erklärt, welche als Sohle für den NAO Roboter eingesetzt werden. Diese Erklärungen basieren auf einem Buch von Pelteret und Steinmann \cite{map2020}, welches für eine weiterführende Lektüre empfehlenswert ist. Danach wird noch kurz auf die Eigenschaften von Carbonyleisenpulver eingegangen, welche in dem MAP eingebettet werden.

In dem Kapitel \ref{aufbau} werden die für diese Arbeit entworfenen und hergestellten Konstruktionen vorgestellt. Dies beginnt mit einem Teil von NAOs Schuh, welcher durch ein 3D gedrucktes Gestell ersetzt wird. Weiterhin wird die Herstellung der MAP Sohle erläutert. Schließlich geht es um die Laufstegkontruktion, welche als Untergrund für die Gangdurchläufe des NAO verwendet wird. 

Die Versuchsdurchführung (siehe Kapitel \ref{durchführung}) gibt einen Einblick der Konstruktionen aus Kapitel \ref{theorie} und \ref{aufbau} sowie über den gesamten Ablauf von der Herstellung bis zur Messung. Außerdem wird hier detailliert auf den Messvorgang eingegangen. 

Schließlich werden die Messergebnisse in Kapitel \ref{auswertung} aufgearbeitet wiedergeben und interpretiert. 
% Kapitelübersicht



%%% Local Variables:
%%% mode: latex
%%% TeX-master: "main"
%%% End: